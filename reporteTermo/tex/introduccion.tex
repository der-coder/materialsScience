\section{Introducci\'on}
La c\'amara termogr\'afica est\'a dise\~nada para permitir al ser humano percibir rangos de
temperatura de diversos objetos sin necesidad de tocarlos o estar en proximidad de ellos.
Sus usos son varios, ya que es posible utilizarlos para:
\begin{itemize}
 \item Detecci\'on de objetos o seres vivos dentro de una regi\'on
 \item Identificar cambios de temperatura importantes en objetos dentro de un per\'iodo de tiempo reducido
 \item Evaluar la situaci\'on de un proceso t\'ermico sin necesidad de tener contacto con la maquinaria
\end{itemize}

Para demostrar el uso y aplicaci\'on de dicho instrumento, se propone una pr\'actica demonstrativa,
en la que se analiza la condici\'on de un segmento de aluminio 6061, despu\'es de ser cortado en el laboratorio
de mec\'anica.

\section{Objetivos}
\begin{itemize}
 \item Aprender a utilizar la c\'amara termogr\'afica
 \item Capturar el comportamiento de un material despu\'es de haber experimentado un corte
 \item Observar y analizar la fotograf\'ia obtenida
\end{itemize}

\section{Descripci\'on y Presentaci\'on}
Para la realizaci\'on de la pr\'actica fue necesario contactar al ingeniero Andr\'es L\'opez, quien se encuentra en
la sala Terra. Se solicit\'o el material, y se procedi\'o al laboratorio de mec\'anica para analizar la muestra.\\

La muestra se obtuvo de cortar un bloque de aluminio 6061 de 10 cm de ancho por 4 cm de largo, dejando un elemento
de un cent\'imetro de espesor. El corte se realiz\'o con una cegueta, provista por el personal del laboratorio.