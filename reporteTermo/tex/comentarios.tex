\section{Comentarios, observaciones y conclusiones}
\subsection{Ricardo Reyes Ram\'irez}
\subsubsection{Comentarios y Observaciones}
A traves de la pr\'actica observamos, y realizamos el proceso para un an\'alisis termogr\'afico.
Pudimos confirmar lo observado en la sesi\'on demostrativa, gracias al uso de la c\'amar disponible
en la instituci\'on.

\subsection{Isaac Ayala Lozano}
\subsubsection{Comentarios}
La c\'amara termogr\'afica, a pesar de su reducido tiempo de uso, nos permiti\'o considerar posibles usos
en \'areas diferentes de la materia. Esto debido a que su funcionamiento no se limita a un \'area, es posible
aplicar esta herramienta en nuevos campos de investigaci\'on de la industria de acuerdo a las necesidades y
restricciones de las empresas.

\subsubsection{Observaciones}
Ignorando el hecho de que no se limaron asperezas del metal por necesidad de analizar r\'apidamente el material,
se observa que el corte gener\'o una cantidad considerable de energ\'ia. Dicha energ\'ia es el resultado de la
resistencia del material y de la fricci\'on entre objetos.

\subsubsection{Conclusiones}
La c\'amara t\'ermica permite observar detalles de una pr\'actica o proceso que no se hab\'ian considerado con anterioridad.
En este caso, notamos que es posible ver c\'omo se disipa el calor en el material, lo que tiene una aplicaci\'on muy importante
en situaciones donde mantener una temperatura espec\'ifica es esencial.

\subsection{Alejandro Barrera Pliego}
\subsubsection{Comentarios y Observaciones}
Esta pr\'actica nos sirvi\'o para identificar como la energ\'ia va fluyendo en los distintos cuerpos; en nuestro caso
vemos c\'omo el aluminio absorbi\'o toda la energ\'ia de la cegueta. \'Esto nos sirvi\'o para entender de manera pr\'actica
lo que es la transferencia de energ\'ia por calor y fricci\'on.

\subsection{Rafael Moreno Castro}
\subsubsection{Comentarios y Observaciones}
Esta pr\'actica nos fue útil para comprender las relaciones entre los temas vistos en
clase y su aplicaci\'on dentro de la empresa. Nos ayud\'o a ampliar nuestros conocimientos
y experiencias, tanto individualmente como equipo. Al final me encuentro satisfecho con
el resultado obtenido

\subsection{Augusto Rodr\'iguez Nolasco}
\subsubsection{Comentarios y Observaciones}
Pudimos comprender mejor el comportamiento de los materiales de acuerdo a la temperatura que presentan, y
que \'estas pueden variar dependiendo de las condiciones ambientales a las que son sometidos.