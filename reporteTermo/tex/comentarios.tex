\section{Comentarios, observaciones y conclusiones}
\subsection{Ricardo Reyes Ram\'irez}
\subsubsection{Comentarios y Observaciones}

\subsubsection{Conclusiones}

\subsection{Isaac Ayala Lozano}
\subsubsection{Comentarios}

\subsubsection{Observaciones}

\subsubsection{Conclusiones}

\subsection{Alejandro Barrera Pliego}
\subsubsection{Comentarios y Observaciones}

\subsubsection{Conclusiones}

\subsection{Rafael Moreno Castro}
\subsubsection{Comentarios y Observaciones}
Esta pr\'actica nos fue útil para comprender las relaciones entre los temas vistos en
clase y su aplicaci\'on dentro de la empresa. Nos ayud\'o a ampliar nuestros conocimientos
y experiencias, tanto individualmente como equipo. Al final me encuentro satisfecho con
el resultado obtenido

\subsection{Augusto Rodr\'iguez Nolasco}
\subsubsection{Comentarios y Observaciones}
La pr\'actica de un análisis metalográfico nos permitió comprender mejor la estructura interna
de un acero, saber los pasos necesarios para poder llevar a cabo éste y ver de otra forma distinta
los materiales; ya que con este análisis pudimos observar la estructura interna de un tipo de acero,
que fue en este caso la forma de los granos que componen dicho material. De manera similar sabemos
que tipos de pruebas se pueden hacer a los materiales para poder conocerlos más a fondo y obtener más
y mejores resultados u observaciones de las ya conocidas.