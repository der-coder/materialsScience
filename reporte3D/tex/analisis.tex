\section{An\'alisis de resultados}
Al analizar el resultado de cada escaneo, es posible observar las lmitaciones de las herramientas.
Para el caso de los lentes de seguridad, fue imposible obtener una geometr\'ia \'util. Solamente
se puedieron rescatar partes del armaz\'on. No obstante, el resultado fue bastante bueno considerando que
fue la primer prueba del esc\'aner.\\
El escaneo del flex\'ometro fue mucho m\'as exitoso, pudiendo visualizar el contorno del mismo en su mayor\'a.
No obstante, este modelo tambi\'en tuvo sus fallas ya que secciones de la pieza no fueron registradas por
la herramienta.\\

Se puede notar que la calidad del escaneo y su precisi\'on dependen en gran medida del material, su composici\'on y
su forma. Ya que para el objeto opaco y m\'as alto, fue m\'as f\'acil obtener un resultado de calidad.