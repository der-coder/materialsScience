\section{Comentarios, observaciones y conclusiones}
\subsection{Ricardo Reyes Ram\'irez}
\subsubsection{Comentarios y Observaciones}
A traves de la pr\'actica observamos, y realizamos dos escaneos. Resolvimos algunas dudas que hab\'ian quedado
de la sesi\'on informativa, al encargarnos nosotros de la actividad en tu totalidad.

\subsection{Isaac Ayala Lozano}
\subsubsection{Comentarios}
Los resultados del escaneo no fueron los esperados, esto debido a que no se conoc\'ian las limitaciones del esc\'aner.

\subsubsection{Observaciones}
Se descubri\'o que el esc\'aner tiene gran dificultad para detectar superficies cristalinas, reflejantes o transl\'ucidas.
Esto ocasion\'o la necesidad de conseguir otros objetos para escanear.

\subsubsection{Conclusiones}
A pesar de las limitaciones que la herramienta posee, es posible reducir los tiempos de desarrollo de nuevos objetos
al utilizar esta t\'ecnica para poder generar geometr\'ia \'util.

\subsection{Alejandro Barrera Pliego}
\subsubsection{Comentarios y Observaciones}
El escaner 3D nos permiti\'o comprender de mejor manera el uso de esta herramienta que en un futuro nos ser\'a
muy \'util para nuestros labores.

\subsection{Rafael Moreno Castro}
\subsubsection{Comentarios y Observaciones}
Esta pr\'actica nos fue útil para comprender las relaciones entre los temas vistos en
clase y su aplicaci\'on dentro de la empresa. Nos ayud\'o a ampliar nuestros conocimientos
y experiencias, tanto individualmente como equipo. Al final me encuentro satisfecho con
el resultado obtenido.

\subsection{Augusto Rodr\'iguez Nolasco}
\subsubsection{Comentarios y Observaciones}
Como ya era conocido, pueden obtenerse modelos 3D en computadora a partir de piezas o prototipos f\'isicos cuando
sean dif\'iciles de ser modelados desde cero. \'Unicamente haciendo una serie de referencias de diferentes puntos
a lo largo del modelo deseado. El es\'aaner puede leer y crear la geometr\'ia de dicha pieza, y posteriormente usarlo
para trabajar como mejor convenga al usuario.