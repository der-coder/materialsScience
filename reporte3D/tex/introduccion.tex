\section{Introducci\'on}
Existen geometr\'ias que diversos productos poseen que son muy dif\'iciles de reproducir con
t\'ecnicas tradicionales. Como respuesta a esta limitaci\'on, surge la tecnolog\'ia de escaneo 3D.
Esta tecnolog\'ia sirve para obtener informaci\'on sobre la supercie de un objeto, con la cual es capaz de recrear
con diferentes grados de exactitud, la geometr\'ia de manera virtual.\\

Sin embargo, esta geometr\'ia no posee informaci\'on \'util para editar, es solamente un archivo STL.
Este tipo de archivo solamente describe la superficie, es incapaz de especificar datos como espesor o
curvas b\'asicas.

\section{Objetivos}
\begin{itemize}
 \item Comprender el proceso para el escaneo tridimensional de diferentes piezas
 \item Evaluar la calidad de la geometr\'ia recuperada
\end{itemize}

\section{Descripci\'on y Presentaci\'on}
Se realizaron dos escaneos para piezas diferentes: un flex\'ometro y un par de lentes de seguridad.
La primer prueba emple\'o el par de lentes, el cual present\'o limitaciones en su escaneo debido a que
el lente es transl\'ucido y el esc\'aner es incapaz de identificarlo. La segunda prueba emple\'o el flex\'ometro,
el cual al ser opaco facilit\'o la recuperaci\'on de sus superficies.\\

Para preparar los objetos, fue necesario cubrirlos con puntos rastreadores de manera distribuida sobre toda la superficie.
A pesar de las instrucciones que las cajas poseen de no reutilizar dichos puntos, se opt\'o por usar aquellos que se encontraban
pegados en el exterior de la caja. La raz\'on de esto es que no se consider\'o vital tener un escaneo perfecto de la piezas.
Adem\'as, se busc\'o no afectar a la instituci\'on con la necesidad de comprar m\'as material para el esc\'aner 3D.